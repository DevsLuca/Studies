\documentclass[11pt]{article}
\usepackage{amsmath, amssymb, amsthm}
\usepackage{geometry}
\usepackage{hyperref}
\usepackage{array}
\usepackage{float}

\geometry{a4paper, margin=1in}

\theoremstyle{definition}
\newtheorem{definition}{Definition}[section]
\newtheorem{example}{Example}[section]

\theoremstyle{plain}
\newtheorem{theorem}{Theorem}[section]
\newtheorem{corollary}{Corollary}[section]
\theoremstyle{plain}
\newtheorem{lemma}{Lemma}[section]
\usepackage{hyperref}
\hypersetup{
    colorlinks=true,
    linkcolor=blue,
    filecolor=magenta,      
    urlcolor=cyan,
}


\title{Chapter 7 --- Independent Sets and Cliques \\}
\author{Luca De Vecchis}
\date{Jan 2026}

\begin{document}

\maketitle
\tableofcontents
\newpage

\section{Independent Sets}

\begin{definition}[Independent Set]
An \textbf{independent set} (or stable set) in a graph $G = (V,E)$ is a set of vertices $I \subseteq V$ such that no two vertices in $I$ are adjacent:
\[
\forall u, v \in I, \ (u,v) \notin E
\]
\end{definition}

\subsection*{Key Concepts}
\begin{itemize}
    \item \textbf{Maximum Independent Set:} An independent set of largest possible size, denoted $\alpha(G)$.
    \item \textbf{Maximal Independent Set:} An independent set that cannot be enlarged by including any other vertex.
    \item \textbf{Relationship to Cliques:} An independent set in $G$ corresponds to a clique in the complement graph $\overline{G}$.
\end{itemize}

\begin{example}
Consider a triangle graph $G$ with vertices $\{A, B, C\}$ and edges $\{(A,B),(B,C),(C,A)\}$. The independent sets are:
\[
\{A\}, \{B\}, \{C\}
\]
The maximum independent set size is $\alpha(G) = 1$.
\end{example}

\subsection*{Properties}
\begin{itemize}
    \item Every graph has at least one independent set (possibly of size 1).
    \item For a graph with $n$ vertices and maximum degree $\Delta$:
    \[
    \alpha(G) \ge \frac{n}{\Delta + 1}
    \]
\end{itemize}

\section{Ramsey's Theorem}

\begin{theorem}[Ramsey's Theorem]
For any positive integers $r,s$, there exists a minimum number $R(r,s)$ such that any graph with at least $R(r,s)$ vertices contains either:
\begin{itemize}
    \item A clique of size $r$, or
    \item An independent set of size $s$.
\end{itemize}
\end{theorem}

\subsection*{Recursive Inequality}
\[
R(l,k) \le R(l-1,k) + R(l,k-1)
\]
\begin{itemize}
    \item This allows calculation of upper bounds recursively.
\end{itemize}

\subsection*{Base Cases}
\begin{itemize}
    \item $R(1,k) = 1$ and $R(l,1) = 1$ for any $l,k$.
    \item $R(2,2) = 2$: Any graph with 2 vertices contains either an edge (2-clique) or an independent set of size 2.
\end{itemize}

\subsection*{Examples}
\begin{itemize}
    \item $R(3,3) = 6$: Any graph with 6 vertices contains either a triangle (3-clique) or an independent set of size 3.
    \item $R(4,3) = 9$: Any graph with 9 vertices has either a 4-clique or an independent set of size 3.
\end{itemize}

The following table shows all Ramsey numbers $r(k, l)$ known to date.

\begin{center}
\begin{tabular}{c|ccccccc}
    \diagbox{$k$}{$l$} & 1 & 2 & 3 & 4 & 5 & 6 & 7 \\ \hline
    1 & 1 & 1 & 1 & 1 & 1 & 1 & 1 \\
    2 & 1 & 2 & 3 & 4 & 5 & 6 & 7 \\
    3 & 1 & 3 & 6 & 9 & 14 & 18 & 23 \\
    4 & 1 & 4 & 9 & 18 & & & \\
\end{tabular}
\end{center}

\subsection*{Applications}
\begin{itemize}
    \item Extremal graph theory
    \item Combinatorial designs
    \item Network reliability and computer science
\end{itemize}

\section{Turan's Theorem}

\begin{theorem}[Turan's Theorem]
Let $G$ be a graph with $n$ vertices that contains no $(r+1)$-clique. Then the maximum number of edges in $G$ is attained by the \textbf{Tur\'an graph} $T(n,r)$, which is the complete $r$-partite graph with partition sizes as equal as possible. Formally:
\[
\text{ex}(n, K_{r+1}) = |E(T(n,r))|
\]
\end{theorem}

\subsection*{Intuition}
- The Tur\'an graph $T(n,r)$ distributes vertices into $r$ independent sets, connecting every vertex to all vertices in the other parts.  
- This construction maximizes edges while avoiding a $(r+1)$-clique because any clique larger than $r$ would require two vertices in the same independent set, which are not connected.  

\subsection*{Exact Edge Count with Partitions}
Let $n = rq + s$, with $0 \le s < r$. Partition $n$ vertices into $r$ parts:  
\[
r-s \text{ parts of size } q, \quad s \text{ parts of size } q+1
\]
Then the number of edges in $T(n,r)$ is:
\[
|E(T(n,r))| = \binom{n}{2} - \sum_{i=1}^{r} \binom{n_i}{2} 
\]
where $n_i$ are the sizes of the independent sets (partitions).  
This formula subtracts the missing edges inside each independent set from the total edges of a complete graph.

\subsection*{Approximation for Large $n$}
If all partitions are equal (\(n_i \approx n/r\)):
\[
|E(T(n,r))| \approx \left(1 - \frac{1}{r}\right) \frac{n^2}{2}
\]

\subsection*{Examples}

\begin{example}[Small Graph: $n=5$, $r=2$]
- Partition $5$ vertices into $2$ parts: $q=2$, $s=1$  
- Part sizes: $3$ and $2$  
- Number of edges:
\[
|E(T(5,2))| = 2 \cdot 3 + 3 \cdot 2 = 6 \text{ edges} \quad (\text{max without $K_3$})
\]
\end{example}

\begin{example}[Extremal Graph: $C_5$]
- $C_5$ (5-cycle) has $5$ edges and no triangle ($K_3$).  
- While it does not achieve Tur\'an's maximum for $n=5,r=2$, it illustrates the concept of an extremal graph: maximum edges possible without containing a forbidden subgraph (here $K_3$).
\end{example}

\subsection*{Properties of Turán Graphs}
\begin{itemize}
    \item $T(n,r)$ is unique up to isomorphism.
    \item Contains no $(r+1)$-clique by construction.
    \item Maximizes the number of edges for given $n$ and $r$.
    \item Each vertex is connected to all vertices outside its partition.
\end{itemize}

\subsection*{Notes on Extremal Graphs}
- Extremal graphs are those achieving the maximum number of edges while avoiding a particular subgraph.  
- $C_5$ is extremal for avoiding a triangle with a given vertex set and edge constraints.  
- For larger $n$ and higher $r$, Turán graphs provide exact extremal examples for $K_{r+1}$-free graphs.

\subsection*{Visualization Tip}
- Imagine $T(6,3)$: 6 vertices divided into 3 groups of 2, with all possible edges between groups but none inside.  
- This forms a "complete 3-partite graph" and clearly avoids a 4-clique while maximizing edges.


\section{Applications}

\subsection{Schur's Theorem}

\begin{theorem}[Schur's Theorem]
For any positive integer $r$, there exists a minimum number $S(r)$ such that if the integers $\{1,2,\dots,S(r)\}$ are colored with $r$ colors, there always exists a monochromatic solution to:
\[
x + y = z
\]
where $x, y, z$ are all assigned the same color.
\end{theorem}

\subsubsection*{Discussion}
- $S(r)$ is the smallest integer guaranteeing a monochromatic solution under any $r$-coloring.  
- Schur's theorem is an early result in **Ramsey-type additive combinatorics**, linking number theory and graph theory.  

\subsubsection*{Example: $r=2$}
- Color the integers $\{1,2,3,4,5\}$ with two colors: Red and Blue.  
- No matter how we assign colors, there will always be a monochromatic solution to $x + y = z$.  
- Example coloring:  
\[
\text{Red: } \{1,4\}, \quad \text{Blue: } \{2,3,5\}
\]  
- Monochromatic solution: $2 + 3 = 5$ (all Blue).  
- Here, $S(2) = 5$.

\subsubsection*{Graph-theoretic Connection}
- Consider a graph where:
    \begin{itemize}
        \item Vertices represent integers $1,2,\dots,S(r)$.
        \item An edge connects two vertices $x$ and $y$ to the vertex $z = x+y$.
        \item Coloring integers corresponds to coloring vertices. A monochromatic triangle represents a solution $x+y=z$ in a single color.
    \end{itemize}
- Schur’s theorem can thus be viewed as a **Ramsey-type problem** on graphs with vertex sums forming "triangles."

\subsection{Applications to Geometry}

Many geometric problems can be formulated using **independent sets and cliques**.  

\begin{example}[No Three Points Collinear]
- Problem: Given a set of $n$ points in the plane, find the largest subset in which no three points are collinear.  
- Graph-theoretic formulation:
    \begin{itemize}
        \item Vertices = points in the plane
        \item Edges = pairs of points that form a line with a third point (collinear)
        \item An independent set = set of points with no three collinear
    \end{itemize}
- Using **Turán-type bounds**:
    \[
    \text{max size of independent set } \alpha(G) \ge \frac{n^2}{2m + n} 
    \]
    where $m$ = number of edges (pairs lying on lines with a third point).
- **Ramsey numbers** can be used to guarantee a subset of points in general position or a collinear subset depending on point arrangement.
\end{example}

\begin{example}[Convex Polygon Problem]
- Problem: Determine the largest convex polygon formed from a set of points in general position.  
- Model using **cliques in a geometric graph**:  
    - Vertices = points  
    - Edges connect points forming sides of a convex polygon  
- Ramsey-type arguments guarantee either a large convex polygon (clique) or a large independent set of points not forming convex subsets.
\end{example}

\subsubsection*{Summary of Geometric Applications}
\begin{itemize}
    \item Independent sets model sets avoiding "forbidden configurations" (e.g., collinear points).  
    \item Cliques model desirable geometric patterns (e.g., convex polygons).  
    \item Extremal and Ramsey-type theorems provide **upper/lower bounds** for these problems.  
\end{itemize}


\section{Summary Table}

\begin{center}
\begin{tabular}{| m{3.5cm} | m{5cm} | m{2.5cm} | m{3.5cm} |}
\hline
\textbf{Concept} & \textbf{Definition} & \textbf{Key Parameters} & \textbf{Applications} \\
\hline
Independent Set & Set of vertices with no edges among them & Maximum size: $\alpha(G)$ & Scheduling, network design \\
\hline
Clique & Set of vertices all mutually connected & Maximum size: $\omega(G)$ & Social network analysis \\
\hline
Ramsey's Theorem & Guarantees clique or independent set in large graphs & $R(r,s)$ & Extremal combinatorics, CS \\
\hline
Turan's Theorem & Max edges avoiding $K_{r+1}$ & $T(n,r)$ & Extremal graph theory \\
\hline
Schur's Theorem & Monochromatic sum exists in integer coloring & $S(r)$ & Number theory, combinatorics \\
\hline
Geometry Problem & Independent sets in geometric configurations & -- & Combinatorial geometry \\
\hline
\end{tabular}
\end{center}

\end{document}
